\documentclass{article}
\usepackage{pstricks}
\usepackage{pst-optexp} % 导入pst-optexp宏包
\begin{document}
\begin{pspicture}(10,6) % 定义画布大小
  \psset{unit=0.5cm} % 设置单位为0.5厘米
  \pnode(2,3){A} % 定义物镜的左端节点A
  \pnode(8,3){B} % 定义目镜的右端节点B
  \lens[lens=4 4 2, n=1.5](A)(B) % 在AB之间画一个物镜,曲率半径为4,折射率为1.5
  \lens[lens=2 6 2, abspos=4, n=1.5](A)(B) % 在AB之间距离A点4个单位处画一个目镜,曲率半径为2和6,折射率为1.5
  \optbox[position=start, optboxwidth=1.5, labeloffset=0](A)(B){物体} % 在A点处画一个宽度为1.5的物体
  \optbox[position=end, optboxwidth=1.5, labeloffset=0](A)(B){眼睛} % 在B点处画一个宽度为1.5的眼睛
  \drawwidebeam[beamwidth=0.5, fillstyle=solid, fillcolor=green!20](A){1}{2}{3} % 从物体发出一束宽度为0.5的光线,经过两个透镜后到达眼睛
\end{pspicture}
\end{document}