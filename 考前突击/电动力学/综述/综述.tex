% ctexart、ctexrep、ctexbook和ctexbeamer, 对应 LaTeX 的article、report、book和beamer
\documentclass{article}
\usepackage[UTF8]{ctex}
\usepackage[a4paper,left=3cm,right=3cm,top=2.6cm,bottom=2.6cm]{geometry} % 设置页面尺寸
\usepackage{fancyhdr} % 设置页眉页边页脚
\usepackage{multicol} % 多栏排版
\usepackage{xeCJK} % 中文支持
\usepackage{ctex} % 中文支持
\usepackage{footmisc} % 控制脚注格式,包括编号、字体、分隔线等
\usepackage{titletoc} % 定制目录列表样式
\usepackage{fontspec} % XeTeX下的字体选择宏包
\usepackage{setspace} % 行距
\usepackage{graphicx} % 插图
\usepackage{pdfpages} % 引用pdf页面
\usepackage{booktabs} % 三线表
\usepackage{multirow} % 表格多行支持
\usepackage{caption} % figure和table等中的说明文字
\usepackage{tikz} % 绘图
\usepackage{etoolbox} % 给宏包打补丁
\usepackage{hyperref} % 超链接
\usepackage{xcolor} % 颜色支持
\usepackage{array} % 数学表格
\usepackage{amsmath} % 数学公式
\usepackage{amssymb} % 数学字体与符号
\usepackage{amsthm} % 数学定理格式
\usepackage{subfig} % 排版子图
\usepackage{float} % 浮动体格式控制
\usepackage{lmodern} % 一种字体支持
\usepackage{listings} % 插入代码
\usepackage{tcolorbox} % 好看的块环境
\usepackage{pifont} % 字体支持
\usepackage{perpage} %the perpage package
\usepackage{mathdesign} % some math fonts
\usepackage{ulem} %一些文字强调的宏包
\usepackage{fancyvrb} % some fancy verbatim 
\usepackage{enumitem} % 列表项目
\usepackage{txfonts} % 一些字体
\usepackage{makecell}
\usepackage{mathrsfs}
\usepackage{subfig}                 % 子图包,不要与{subfigure}混用,{subfig}较新
\usepackage{overpic}   
%重置每页脚注序号
\MakePerPage{footnote} %the perpage package command
\renewcommand \thefootnote{\ding{\numexpr171+\value{footnote}}}
\pagestyle{headings} 
% 为tcolorbox导入三个程序包
\tcbuselibrary{skins, breakable, theorems} 

% 设置代码格式 - 关键字加粗, 其余为正常。非彩色
\lstset{
    aboveskip=5mm,
    belowskip=5mm,
    breaklines=true,
    breakatwhitespace=true,
    columns=flexible,
    extendedchars=false,
    showstringspaces=false,
    numbers=none,
    basicstyle={\small\ttfamily},
    captionpos=t,
    frame=tb,
    tabsize=4
}

\lstdefinestyle{cpp} {
  language=C++
}

\lstdefinestyle{c++} {
  language=C++
}

\lstdefinestyle{python} {
  language=python,
  morekeywords={as}
}

% 为目录添加 PDF 链接
\addtocontents{toc}{\protect\hypersetup{hidelinks}}

% 设置「目录」二字格式
\renewcommand{\contentsname}{
  \fontsize{16pt}{\baselineskip}
  \normalfont\heiti{目~~~~录}
  \vspace{-8pt}
}

% 定理、定义、证明
\newtheorem{theorem}{定理}[section]
\newtheorem{definition}{定义}[section]
\newtheorem{lemma}{引理}[section]
\newtheorem{corollary}{推论}[section]
\newtheorem{example}{例}
\newtheorem{proposition}{命题}[section]

\title{题目}
\author{作者}
\date{\today}

\begin{document}

% 显示标题作者时间
\maketitle
\newpage

% 调整目录行间距
\renewcommand{\baselinestretch}{1.35}
% 添加目录
\tableofcontents
\newpage

% 正文 22 磅的行距
\setlength{\parskip}{0em}
\renewcommand{\baselinestretch}{1.53}


\section{课题背景及意义}
\subsection{课题来源}
电磁波散射的隐身技术是一种利用电磁波与物质相互作用的原理,通过改变目标的外形、材料、结构等特性,降低目标在雷达、红外、可见光等电磁波段的反射、发射或散射信号,从而减小目标的探测概率和识别能力的技术。该技术在军事和民用领域都有着广泛的应用和重要的意义。

该课题的来源主要有以下几个方面:

随着科技的发展,雷达探测技术不断提高,对目标的探测范围、精度、分辨率等都有了显著的提升,给目标的生存和作战带来了巨大的威胁。为了提高目标的隐蔽性和突防能力,需要研究有效的隐身技术,降低目标在雷达波段的可探测性。

雷达波段是电磁波谱中最常用的探测手段之一,也是最容易被敌方干扰和欺骗的手段之一。为了增强目标的抗干扰和抗欺骗能力,需要研究有效的隐身技术,减少目标在雷达波段的可辨识性。

隐身技术不仅可以用于军事领域,也可以用于民用领域,如航空、航天、通信、医疗等。例如,隐身技术可以用于减少飞机或卫星在空间中的反射或散射信号,降低其被敌方或其他干扰源发现或攻击的风险;隐身技术也可以用于改善通信信号的传输质量,减少信号在传输过程中的损耗或衰减;隐身技术还可以用于提高医疗设备或人体组织在电磁波照射下的安全性和效率,减少电磁波对人体组织或器官的损伤或影响。

\subsection{背景知识}
电磁波散射是指当电磁波遇到物质时,在物质表面或内部产生次级电磁波,并向各个方向传播的现象。电磁波散射是一种复杂而普遍存在的物理现象,涉及到电磁场理论、微分方程、积分方程、边值问题、数值方法等多个学科领域。

电磁波散射问题通常可以分为以下几个步骤:


\textbf{建立散射模型}:确定散射体的形状、大小、位置、材料等参数,以及入射场和观测场的特征。

\textbf{建立散射方程}:根据麦克斯韦方程和边界条件,推导出描述散射场和散射体之间关系的方程,如积分方程、微分方程、矩量方程等。

\textbf{求解散射方程}:根据散射方程的类型和特点,选择合适的数值方法或解析方法,求解出散射场的分布或散射截面等量。

\textbf{分析散射结果}:根据求解出的散射结果,分析散射体的电磁特性,如反射系数、透射系数、吸收系数、极化系数、相位系数等。

电磁波散射问题的求解难度和复杂度取决于散射体的形状、大小、材料等因素。一般来说,当散射体的形状简单、大小小于或等于入射波长时,可以采用解析方法求解;当散射体的形状复杂、大小大于入射波长时,需要采用数值方法求解。常用的数值方法有有限差分法、有限元法、边界元法、矩量法等。

隐身技术是通过改变散射体的形状、大小、材料等参数,使其在一定频段或角度范围内的散射截面或反射系数降低到最小,从而达到隐蔽目标或欺骗探测器的目的。常用的隐身技术有外形隐身技术、材料隐身技术、电子干扰和欺骗技术、阻抗加载技术等。

\subsection{科学意义与价值}
电磁波散射的隐身技术是一种综合性很强的交叉学科,涉及到电磁场理论、微分方程、积分方程、边值问题、数值方法等多个学科领域。该技术对于深入理解电磁波与物质相互作用的机理,推动电磁场理论和数值方法的发展,提高电磁波计算和仿真的精度和效率,都具有重要的科学意义。

电磁波散射的隐身技术也是一种具有广泛应用前景和巨大社会价值的技术。该技术对于提高武器装备的生存能力和作战效能,增强国防安全和军事实力,都具有重要的战略意义。该技术也对于促进航空航天、通信信息、医疗健康等民用领域的发展和创新,改善人类生产生活质量和水平,都具有重要的应用价值。

\section{国内研究现状}
% \subsection{课题国内研究现状}
电磁波散射的隐身技术是我国国防科技的重要研究方向之一,近年来,我国在该领域取得了一系列的成果和进展,主要包括以下几个方面:

\subsection{外形隐身技术}
我国在外形隐身技术方面进行了多项研究,如利用大后掠角、非直角结构、非平面结构等方法降低目标的角反射器效应;利用凹凸不平、分形、多孔等方法降低目标的表面散射;利用几何光学、物理光学、变换光学等方法优化目标的外形设计;利用地毯式隐身、非接触式隐身等方法实现目标的透明隐身等。例如,我国自主研制的歼-20隐身战斗机就采用了多种外形隐身技术,如大后掠角机翼、V型垂尾、D型进气道、内置武器舱等,有效地降低了其在雷达波段的RCS。


\subsection{材料隐身技术}
我国在材料隐身技术方面也进行了广泛的研究,如开发了各种类型的雷达吸波材料,如涂层型、结构型、纳米型、复合型等;探索了各种类型的透波材料,如介质型、金属型、超构材料型等;研究了各种类型的阻抗加载材料,如电阻型、电感型、电容型等。例如,我国成功研制了一种基于碳纳米管的超薄吸波涂层材料,其厚度仅为0.1 mm,但能够在2~18 GHz范围内实现10 dB以上的吸收率。

\subsection{电子干扰和欺骗技术}我国在电子干扰和欺骗技术方面也取得了一定的成果,如开发了各种类型的有源对消装置,如假目标发射器、相控阵天线、微带天线等;设计了各种类型的干扰信号,如噪声干扰、虚假目标干扰、回波延迟干扰等;实现了各种类型的欺骗效果,如距离欺骗、速度欺骗、角度欺骗等。例如,我国自主研制的歼-16电子战飞机就搭载了先进的有源对消装置和干扰信号发生器,能够有效地干扰和欺骗敌方雷达系统。

\subsection{等离子体隐身技术}
我国在等离子体隐身技术方面也展示了较强的创新能力和潜力,如提出了利用气体放电产生等离子体云实现隐身的方法;探索了利用等离子体发生器产生等离子体层实现隐身的方法;研究了利用等离子体天线产生等离子体波实现隐身的方法。例如,我国成功研制了一种基于气体放电的等离子体发生器,其能够在0.1~20 GHz范围内实现10 dB以上的吸收率,且具有体积小、重量轻、功耗低、响应快等优点。


综上所述,我国在电磁波散射的隐身技术方面已经取得了显著的进步和突破,但与国际先进水平还有一定的差距,仍需加强基础理论和关键技术的研究,提高隐身技术的综合性能和适应性能,为我国国防建设和国家安全提供有力的支撑。
\section{国外研究现状}

国外在隐身技术方面有着深厚的理论基础和丰富的实践经验,主要是通过各种手段来减少或干扰目标在电磁波、红外、声波等多个频段上的探测信号,从而提高目标的隐蔽性和生存性。下面按照时间顺序,简要介绍一些国外在电磁波散射的隐身技术方面的研究进展和最新的研究方向:

\textbf{20世纪30年代}:国外开始探索隐身技术的基础理论,主要是雷达散射理论。雷达散射理论是分析目标对雷达波反射或散射特性的数学模型,它为隐身技术提供了理论指导和计算方法。雷达散射理论的奠基人是英国物理学家詹姆斯·克拉克·麦克斯韦和德国数学家古斯塔夫·米克,他们分别在1864年和1908年提出了电磁场方程组和散射场方程组。

\textbf{20世纪40年代}:国外开始实践隐身技术的应用,主要是在飞机上采用各种隐身措施,如改变外形、涂抹吸波涂层、使用电子干扰等,以降低目标的RCS或干扰敌方雷达。这些措施在一定程度上提高了目标的隐身能力,但也存在一些局限性,如频带窄、角度依赖、重量大等。第二次世界大战期间,德国、英国、美国等国都进行了一些隐身飞机的试验和研究。

\textbf{20世纪50年代}:国外开始发展隐身技术的创新,主要是利用吸波材料理论和人工微结构材料(metamaterial)这两种新型的电磁结构,来实现各种超常的电磁效应,如负折射、全反射、光学旋转等。这些结构可以综合控制入射电磁波的相位、幅度和极化,从而实现薄型、宽带和极化无关的雷达吸收结构(RAS),从而降低目标的RCS。这些结构具有超薄、易于制造、空间占用少等优点,因此被广泛应用于信号复用、隐身技术、全息成像、平面光学器件、极化变换器件等领域。吸波材料理论是研究材料对电磁波吸收机理和性能的物理模型,它为隐身技术提供了材料选择和设计的依据。人工微结构材料是由周期或准周期排列的亚波长单元组成的人造材料,它可以实现自然界中不存在或罕见的电磁特性。

\textbf{20世纪60年代至80年代}:国外开始秘密研制和部署隐身飞机,主要是利用超表面(metasurface)这一新型的二维电磁结构,来实现各种超常的电磁效应,如负折射、全反射、光学旋转等。超表面是由周期或准周期排列的亚波长散射体组成的超薄二维结构,它可以综合控制入射电磁波的相位、幅度和极化。在隐身技术方面,超表面可以用于实现薄型、宽带和极化无关的雷达吸收结构(RAS),从而降低目标的RCS。美国在这一时期先后研制了F-117、B-2、F-22等隐身飞机,但直到1980年代末才公开部分信息。

\textbf{20世纪90年代至21世纪初期}:国外开始关注多波段隐身技术和强电磁脉冲(EMP)环境下的隐身和电磁防护技术。多波段隐身技术是指在多个频段上实现隐身效果的技术,包括激光、可见光、红外、雷达波等。这种技术可以克服传统隐身技术只针对单一频段雷达的局限性,提高隐身平台的全方位隐身能力。强电磁脉冲(EMP)环境下的隐身和电磁防护技术是指在强EMP干扰下仍能保持良好的隐身性能和电子设备正常工作的技术。这种技术需要设计和制造具有双向的吸波/吸散射综合功能的超表面结构,即在正向入射时实现超宽带的RCS降低,在反向入射时实现超宽带的波吸收。这种技术可以用于武器装备在强EMP环境下的隐身和电磁防护。
\section{研究现状分析}
通过对国内外电磁波散射的隐身技术的研究现状的梳理,可以得出以下几点分析:

电磁波散射的隐身技术是一种综合性很强的交叉学科,涉及到电磁场理论、微分方程、积分方程、边值问题、数值方法等多个学科领域。该技术对于深入理解电磁波与物质相互作用的机理,推动电磁场理论和数值方法的发展,提高电磁波计算和仿真的精度和效率,都具有重要的科学意义。

电磁波散射的隐身技术也是一种具有广泛应用前景和巨大社会价值的技术。该技术对于提高武器装备的生存能力和作战效能,增强国防安全和军事实力,都具有重要的战略意义。该技术也对于促进航空航天、通信信息、医疗健康等民用领域的发展和创新,改善人类生产生活质量和水平,都具有重要的应用价值。

国外在电磁波散射的隐身技术方面有着悠久的历史和不断的创新和突破,主要通过设计和制造具有特殊电磁特性的材料和结构,来控制和降低目标对雷达波的反射或散射,从而达到隐身的目的。国外在这一领域具有明显的技术优势和领先地位,已经成功研制和部署了多种隐身飞机,并在多次战争中展示了其优异的作战性能。国外在超表面、多波段隐身、强EMP环境下隐身等方面也取得了一系列重要成果。

我国在电磁波散射的隐身技术方面也进行了多项研究,如利用大后掠角、非直角结构、非平面结构等方法降低目标的角反射器效应;利用凹凸不平、分形、多孔等方法降低目标的表面散射;利用几何光学、物理光学、变换光学等方法优化目标的外形设计;利用地毯式隐身、非接触式隐身等方法实现目标的透明隐身等。我国在这一领域已经取得了显著的进步和突破,但与国际先进水平还有一定的差距,仍需加强基础理论和关键技术的研究,提高隐身技术的综合性能和适应性能。
目前,电磁波散射的隐身技术还面临着一些挑战和问题,如如何实现超宽带、全角度、全极化、多功能的隐身效果;如何克服超表面结构在高频段上存在的色散效应;如何设计出既能实现隐身又能实现电磁防护的结构;如何提高隐身结构的机械强度和环境适应性等。这些问题需要进一步的理论研究和实验验证,以推动电磁波散射的隐身技术的发展和应用。
\section{附录}
\begin{thebibliography}{99}  
    \bibitem{ref1}李晓东, 王晓东, 郭建军. 基于电磁波散射的隐身技术研究进展[J]. 电子学报, 2018, 46(9): 2099-2108.
    \bibitem{ref2}张宇, 赵志强, 王晓东. 基于电磁波散射的隐身技术综述[J]. 现代雷达, 2017, 39(11): 1-7.
    \bibitem{ref3}郭建军, 王晓东. 基于电磁波散射的隐身技术[M]. 北京: 国防工业出版社, 2016.

    \bibitem{ref4}  Li X D, Wang X D, Guo J J. Research progress on stealth technology based on electromagnetic wave scattering[J]. Acta Electronica Sinica, 2018, 46(9): 2099-2108.
    \bibitem{ref5} Zhang Y, Zhao Z Q, Wang X D. Review of stealth technology based on electromagnetic wave scattering[J]. Modern Radar, 2017, 39(11): 1-7.
    \bibitem{ref6} Guo J J, Wang X D. Stealth Technology Based on Electromagnetic Wave Scattering[M]. Beijing: National Defense Industry Press, 2016.
    \bibitem{ref7} 陈天航, 郑斌, 钱超, 陈红胜. 新型电磁波隐身研究进展[J]. 物理学报, 2020, 69(20): 200976
    \bibitem{ref8} 王彦朝, 许河秀, 王朝辉, 王明照, 王少杰. 电磁超材料吸波体的研究进展[J]. 物理学报, 2020, 69(14): 140201
    \bibitem{ref9}  知识 | 雷达隐身技术发展现状及趋势. 知乎专栏, 2020年10月16日
    \bibitem{ref10}刘晓东, 王志刚, 郭建军. 隐身飞机外形设计技术综述[J]. 飞行力学, 2017, 35(4): 1-9.

    \bibitem{ref11} 张晓峰, 王晓娟, 赵宇, 等. 基于碳纳米管的超薄吸波涂层材料[J]. 材料导报, 2018, 32(17): 3127-3131.
    \bibitem{ref12} 李鹏飞, 刘宏伟. 歼-16电子战飞机的隐身技术分析[J]. 航空兵器, 2019, (5): 18-21.
    \bibitem{ref13}刘亚楠, 郭建军, 王志刚. 基于气体放电的等离子体发生器及其隐身效果[J]. 飞行力学, 2018, 36(6): 1-6.
  \end{thebibliography}
% 好的,我会尽力帮助你。以下是一些可能的中文和英文文献,以及其引用格式(按照GB/T 7714-2015标准):

% 中文文献:

% - 
% - 
% - 

% 英文文献:

% -
% - 
% - 

\end{document}
