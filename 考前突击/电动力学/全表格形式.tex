% ctexart、ctexrep、ctexbook和ctexbeamer, 对应 LaTeX 的article、report、book和beamer
\documentclass{article}
\usepackage[UTF8]{ctex}
\usepackage[a4paper,left=3cm,right=3cm,top=2.6cm,bottom=2.6cm]{geometry} % 设置页面尺寸
\usepackage{fancyhdr} % 设置页眉页边页脚
\usepackage{multicol} % 多栏排版
\usepackage{xeCJK} % 中文支持
\usepackage{ctex} % 中文支持
\usepackage{footmisc} % 控制脚注格式,包括编号、字体、分隔线等
\usepackage{titletoc} % 定制目录列表样式
\usepackage{fontspec} % XeTeX下的字体选择宏包
\usepackage{setspace} % 行距
\usepackage{graphicx} % 插图
\usepackage{pdfpages} % 引用pdf页面
\usepackage{booktabs} % 三线表
\usepackage{multirow} % 表格多行支持
\usepackage{caption} % figure和table等中的说明文字
\usepackage{tikz} % 绘图
\usepackage{etoolbox} % 给宏包打补丁
\usepackage{hyperref} % 超链接
\usepackage{xcolor} % 颜色支持
\usepackage{array} % 数学表格
\usepackage{amsmath} % 数学公式
\usepackage{amssymb} % 数学字体与符号
\usepackage{amsthm} % 数学定理格式
\usepackage{subfig} % 排版子图
\usepackage{float} % 浮动体格式控制
\usepackage{lmodern} % 一种字体支持
\usepackage{listings} % 插入代码
\usepackage{tcolorbox} % 好看的块环境
\usepackage{pifont} % 字体支持
\usepackage{perpage} %the perpage package
\usepackage{mathdesign} % some math fonts
\usepackage{ulem} %一些文字强调的宏包
\usepackage{fancyvrb} % some fancy verbatim 
\usepackage{enumitem} % 列表项目
\usepackage{txfonts} % 一些字体
\usepackage{makecell}
\usepackage{mathrsfs}
\usepackage{subfig}                 % 子图包,不要与{subfigure}混用,{subfig}较新
\usepackage{overpic}   
%重置每页脚注序号
\MakePerPage{footnote} %the perpage package command
\renewcommand \thefootnote{\ding{\numexpr171+\value{footnote}}}
\pagestyle{headings} 
% 为tcolorbox导入三个程序包
\tcbuselibrary{skins, breakable, theorems} 

% 设置代码格式 - 关键字加粗, 其余为正常。非彩色
\lstset{
    aboveskip=5mm,
    belowskip=5mm,
    breaklines=true,
    breakatwhitespace=true,
    columns=flexible,
    extendedchars=false,
    showstringspaces=false,
    numbers=none,
    basicstyle={\small\ttfamily},
    captionpos=t,
    frame=tb,
    tabsize=4
}

\lstdefinestyle{cpp} {
  language=C++
}

\lstdefinestyle{c++} {
  language=C++
}

\lstdefinestyle{python} {
  language=python,
  morekeywords={as}
}

% 为目录添加 PDF 链接
\addtocontents{toc}{\protect\hypersetup{hidelinks}}

% 设置「目录」二字格式
\renewcommand{\contentsname}{
  \fontsize{16pt}{\baselineskip}
  \normalfont\heiti{目~~~~录}
  \vspace{-8pt}
}

% 定理、定义、证明
\newtheorem{theorem}{定理}[section]
\newtheorem{definition}{定义}[section]
\newtheorem{lemma}{引理}[section]
\newtheorem{corollary}{推论}[section]
\newtheorem{example}{例}
\newtheorem{proposition}{命题}[section]

\title{题目}
\author{作者}
\date{\today}

\begin{document}

% 显示标题作者时间
\maketitle
\newpage

% 调整目录行间距
\renewcommand{\baselinestretch}{1.35}
% 添加目录
\tableofcontents
\newpage

% 正文 22 磅的行距
\setlength{\parskip}{0em}
\renewcommand{\baselinestretch}{1.93}


\section{电磁现象}

\begin{table}[H]
      \centering
      \begin{tabular}{|m{1.5cm}<{\centering}|m{3.5cm}<{\centering}|m{3cm}<{\centering}|m{4.8cm}<{\centering}|m{1.8cm}<{\centering}|}
      \hline
      % \mathrm{R}e,\mathrm{I}m$实部,虚部
          名称 & 公式&其他表示 &说明1 &说明2\\ \hline
          库仑定律 &$\displaystyle \vec{F}=\frac{qq'\vec{r}}{4 \pi \varepsilon_0 r^3} $&$ \vec{F'}=-\vec{F} $&&\\ \hline
          电场 &$\displaystyle \vec{E}=\frac{F}{q'}=\frac{q\vec{r}}{4 \pi \varepsilon_0 r^3} $&&q是源电荷,$q'$是试探电荷&\\ \hline
          高斯定理&
          $\displaystyle \oint_{S}E \cdot d \overrightarrow{s}=\frac{q}{\varepsilon}$
          &&对于闭合曲面内的每一个点电荷,其形成电通量为一个定值& 外面的q形成的为0\\ \hline
          电流密度&&&&\\ \hline
          电流连续性方程&
          $\displaystyle \nabla \cdot \vec{j}+\frac{\partial \rho}{\partial t}=0  $
          &&&\\ \hline
          安培定律&&&&\\ \hline
          -----  &------磁场&--------&---------&------
            \\ \hline 
            磁场力&&&&  \\
            \hline
            静电场的散度&$\displaystyle \nabla \cdot \vec{E_{\text{静}}}=\frac{\rho(\vec{x'})}{\varepsilon}$&
            $\displaystyle \nabla \cdot \vec{E_{\text{动}}}=0$
            &$\vec{E_{\text{总}}}={\vec{E}_{\text{静}}+\vec{E}_{\text{动}}}$&\\ \hline
            静电场的旋度&$\displaystyle \nabla \times \vec{E_{\text{静}}}=0$&
            $\displaystyle \nabla \times \vec{E_{\text{动}}}=-\frac{\partial B}{\partial t}  $
            &&证明请看小册\\ \hline
            静磁场的散度&$\displaystyle \nabla \cdot \vec{B}=0$&&&\\ \hline
            静磁场的旋度&$\displaystyle \nabla \times \vec{B}=\mu_0 \vec{j}(\vec{x})$&&&\\ \hline
            感应电动势&
            $\displaystyle \varepsilon=- \frac{d \phi _{m}}{dt}=- \frac{d}{dt} \int _{S}\overline{B}\cdot d \overrightarrow{s}$
            &
            $\displaystyle \nabla \times \vec{E_{\text{感}}}+\frac{\partial B}{\partial t}=0  $
            &&\\ \hline
           位移电流&&&电流连续性方程和$\nabla \cdot \vec{B}$矛盾&\\ \hline
           麦克斯韦方程&&&&\\ \hline
        \end{tabular}
        \caption{}
      \end{table}
\end{document}
