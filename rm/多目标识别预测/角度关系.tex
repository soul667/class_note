% ctexart、ctexrep、ctexbook和ctexbeamer, 对应 LaTeX 的article、report、book和beamer
\documentclass{article}
\usepackage[UTF8]{ctex}
\usepackage[a4paper,left=3cm,right=3cm,top=2.6cm,bottom=2.6cm]{geometry} % 设置页面尺寸
\usepackage{fancyhdr} % 设置页眉页边页脚
\usepackage{multicol} % 多栏排版
\usepackage{xeCJK} % 中文支持
\usepackage{ctex} % 中文支持
\usepackage{footmisc} % 控制脚注格式,包括编号、字体、分隔线等
\usepackage{titletoc} % 定制目录列表样式
\usepackage{fontspec} % XeTeX下的字体选择宏包
\usepackage{setspace} % 行距
\usepackage{graphicx} % 插图
\usepackage{pdfpages} % 引用pdf页面
\usepackage{booktabs} % 三线表
\usepackage{multirow} % 表格多行支持
\usepackage{caption} % figure和table等中的说明文字
\usepackage{tikz} % 绘图
\usepackage{etoolbox} % 给宏包打补丁
\usepackage{hyperref} % 超链接
\usepackage{xcolor} % 颜色支持
\usepackage{array} % 数学表格
\usepackage{amsmath} % 数学公式
\usepackage{amssymb} % 数学字体与符号
\usepackage{amsthm} % 数学定理格式
\usepackage{subfig} % 排版子图
\usepackage{float} % 浮动体格式控制
\usepackage{lmodern} % 一种字体支持
\usepackage{listings} % 插入代码
\usepackage{tcolorbox} % 好看的块环境
\usepackage{pifont} % 字体支持
\usepackage{perpage} %the perpage package
\usepackage{mathdesign} % some math fonts
\usepackage{ulem} %一些文字强调的宏包
\usepackage{fancyvrb} % some fancy verbatim 
\usepackage{enumitem} % 列表项目
\usepackage{txfonts} % 一些字体
\usepackage{makecell}
\usepackage{mathrsfs}
\usepackage{subfig}                 % 子图包,不要与{subfigure}混用,{subfig}较新
\usepackage{overpic}   
%
\usepackage{eso-pic}
\usetikzlibrary{quotes,angles}
\usetikzlibrary{calc}
\usetikzlibrary{decorations.pathreplacing}
\usepackage{background}
\usepackage{lipsum}
\usepackage{xpatch}
\usepackage{fancyhdr}
\usepackage{tikz-3dplot}
\usetikzlibrary{3d}
\makeatletter
% prepare \bg@text
\def\bg@text{}
\foreach \i in {1, ..., 10} {
  \g@addto@macro\bg@text{古翱翔 2021251124  古翱翔 2021251124 古翱翔 2021251124   古翱翔 2021251124 古翱翔 2021251124\\}
}
\g@addto@macro\bg@text{古翱翔 2021251124    古翱翔 2021251124 古翱翔 2021251124  古翱翔 2021251124 古翱翔 2021251124}

% set contents to \bg@text
\backgroundsetup{contents={\bg@text}}

% allow multiline contents
\xpatchcmd\bg@material
  {inner sep=0pt}
  {inner sep=0pt, align=center, font=\fontsize{1.7}{4}\selectfont}
  {}{\fail}
\makeatother

% \usepackage{draftwatermark}         % 所有页加水印
% %\usepackage[firstpage]{draftwatermark} % 只有第一页加水印
% \SetWatermarkText{古翱翔 2021251124}           % 设置水印内容
% %\SetWatermarkText{\includegraphics{fig/texlion.png}}         % 设置水印logo
% \SetWatermarkLightness{0.85}             % 设置水印透明度 0-1
% \SetWatermarkScale{0.5}   


%重置每页脚注序号
% \MakePerPage{footnote} %the perpage package command
% \renewcommand \thefootnote{\ding{\numexpr171+\value{footnote}}}
\pagestyle{headings} 
% 为tcolorbox导入三个程序包
\tcbuselibrary{skins, breakable, theorems} 

% 设置代码格式 - 关键字加粗, 其余为正常。非彩色
\lstset{
    aboveskip=5mm,
    belowskip=5mm,
    breaklines=true,
    breakatwhitespace=true,
    columns=flexible,
    extendedchars=false,
    showstringspaces=false,
    numbers=none,
    basicstyle={\small\ttfamily},
    captionpos=t,
    frame=tb,
    tabsize=4
}

\lstdefinestyle{cpp} {
  language=C++
}

\lstdefinestyle{c++} {
  language=C++
}

\lstdefinestyle{python} {
  language=python,
  morekeywords={as}
}

% 为目录添加 PDF 链接
\addtocontents{toc}{\protect\hypersetup{hidelinks}}

% 设置「目录」二字格式
\renewcommand{\contentsname}{
  \fontsize{16pt}{\baselineskip}
  \normalfont\heiti{目~~~~录}
  \vspace{-8pt}
}

% 定理、定义、证明
\newtheorem{theorem}{定理}[section]
\newtheorem{definition}{定义}[section]
\newtheorem{lemma}{引理}[section]
\newtheorem{corollary}{推论}[section]
\newtheorem{example}{例}
\newtheorem{proposition}{命题}[section]
\pagestyle{fancy}
\fancyhf{}
\fancyhead[C]{2021251124 古翱翔}
\cfoot{\thepage} 
\title{大物实验}
\author{古翱翔 2021251124}

\date{\today}

\begin{document}

% 显示标题作者时间
\maketitle
\newpage

% 调整目录行间距
\renewcommand{\baselinestretch}{1.35}
% 添加目录
\tableofcontents
\newpage

% 正文 22 磅的行距
\setlength{\parskip}{0em}
\renewcommand{\baselinestretch}{1.53}


\section{第一题}
\tdplotsetmaincoords{50}{125}
        \begin{figure}[H]
            \centering
            % \begin{tikzpicture}[scale=1.5]
            %   \pgfversion
            %   \draw[thick,->] (0,0,0) -- (2.5,0,0) node[anchor=north east]{$x$};
            %   \draw[thick,->] (0,0,0) -- (0,2.5,0) node[anchor=north west]{$y$};
            %   \draw[thick,->] (0,0,0) -- (0,0,2.5) node[anchor=south]{$z$};
            %   \draw[thick,->] (0,0,0) -- (0,0,2.5) node[anchor=south]{$z$};

            % \end{tikzpicture}
            \begin{tikzpicture}
              [tdplot_main_coords,
                cube/.style={very thick,black},
                grid/.style={very thin,gray},
                axis/.style={->,blue,thick}]
   
            % %draw the axes
            \draw[] (0,0,0) -- (0,0,1) node[left]{$l$};
            \draw[axis] (0,0,0) -- (3,0,0) node[anchor=west]{$x$};
            \draw[axis] (0,0,0) -- (0,3,0) node[anchor=west]{$y$};
            \draw[axis] (0,0,0) -- (0,0,3) node[anchor=west]{$z$};
          
            % %draw the top and bottom of the cube
            % \draw[cube] (0,0,0) -- (0,2,0) -- (2,2,0) -- (2,0,0) -- cycle;
            % \draw[cube] (0,0,2) -- (0,2,2) -- (2,2,2) -- (2,0,2) -- cycle;
            
            % %draw the edges of the cube
            \draw[cube] (0,0,0) -- (0,0,2) -- (0,2,2) -- (0,2,0) --(0,0,0);
            
            \draw[->,color=red,-latex,very thick] (0,2,1) -- (0,2,1.2) node[left]{I};%<〈node spec〉%> {%<content%>};;
            \filldraw[] (0,1,1) circle(1pt) node[anchor=north]{$O$};
            % \draw[dashed,thick] (0,1,1) -- (1,1,1) node[above]{d};
            \draw[->,thick,-stealth] (0,1,1) -- (2,1,1)node[right]{$B_{O}=4\times {B}_{\mathrm{I}}$} ;

          \end{tikzpicture}
        \caption{O点示意图}
            \end{figure}
\end{document}
